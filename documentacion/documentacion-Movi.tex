\documentclass[11pt]{scrartcl}

%\usepackage{sectsty}
%\usepackage{graphicx}
\usepackage{listings}
\usepackage{amssymb, amsmath}

% Margins
\topmargin=-0.45in
\evensidemargin=0in
\oddsidemargin=0in
\textwidth=6.5in
\textheight=9.0in
\headsep=0.25in

\title{MoviScript}
\subtitle{Lenguaje especifico de dominio para programar m\'oviles}

\author{ Pablo Dobry }
\date{\today}

\begin{document}
\lstset{language=Haskell}
\maketitle	
\pagebreak

\section{Descripci\'on general}

El lenguaje MoviScript es un lenguaje especifico de dominio (DSL) orientado a la 
programaci\'on de m\'oviles como el recorrido de pequeños autos eléctricos 
programables y robots. Este lenguaje tiene la finalidad de facilitar 
la programación de la movilidad, ofreciendo una sintaxis intuitiva 
y funcionalidades específicas para abordar los desafíos asociados con estos dispositivos.
Este DSL está diseñado para simplificar tareas comunes en la programación del desplazamiento, 
promoviendo un desarrollo eficiente.\newline
Este lenguaje cuenta con comandos clasicos de un lenguaje imperativo, como la posiblidad 
de operar sobre expresiones matem\'aticas y booleanas, condicionales y bucles.
Cuenta con instrucciones espec\'ificas de la programaci\'on de m\'oviles como 
la posiblidad de generar generar una lista de puntos en funci\'on de una expresi\'on 
matem\'atica y una lista de puntos donde evaluarla.\newline
Cuenta con instrucciones para el seguimiento de un camino (una lista de puntos), como as\'i tambi\'en
la posiblidad de que, en caso de verse impedido el avance del m\'ovil elabore 
un plan de contingencia para poder llegar al siguiente punto del camino.\newline
La implementaci\'on de un lenguaje, el DSL se construye directamente embebido 
dentro de Haskell, utilizando sus funcionalidades como base.
Luego, se emplea la biblioteca Parsec de Haskell para llevar a cabo 
el análisis sint\'actico del c\'odigo fuente del DSL.  \newline
Una vez completado el paso de parseo, se procede a la construcción de un 
\'Arbol de Sintaxis Abstracta (AST). Este \'arbol representa 
la estructura jerárquica del c\'odigo fuente y sirve como una 
representaci\'on interna que facilita la manipulaci\'on y evaluación del c\'odigo.\newline
Finalmente, para la evaluación semántica del DSL, se recurre a un evaluador mon\'adico. 
Las m\'onadas en Haskell ofrecen un mecanismo para modelar c\'omputos con efectos 
secundarios, lo que resulta esencial para la interpretación y ejecuci\'on 
de expresiones en nuestro lenguaje.\newline

\pagebreak
\section{Manual de uso}
El lenguaje Movi esta embebido en Haskell por lo que para compilar el lenguaje MoviScript 
se deber\'a tener instalado un compilador de Haskell. Con ghci para su uso se deber\'a 
ejecutar el siguiente comando en la carpeta en la que se encuentran los archivos Main.hs,
Parser.hs y EvalMovi.hs:
\begin{lstlisting}[language=bash]
    ghci Main.hs miPrograma.movi
\end{lstlisting}
De esta forma se ejecuta el programa $miPrograma.movi$. La ejecuci\'on en esta primer versi\'on de 
MoviScript producir\'a una salida en pantalla con las primitivas ejecutadas por el m\'ovil y la traza
 con los puntos por los que pas\'o el m\'ovil.
 Se debe tener instalado ucblogo para poder ejecutar el script en logo con el que se evalua el 
 programa en esta version. Para hacerlo se puede ejecutar en una distribucion Debian o Ubuntu el siguiente comando:
 \begin{lstlisting}[language=bash]
    apt-get install ucblogo
\end{lstlisting}

\section{Expresiones flotantes}
El lenguaje incorpora funciones trigonom\'etricas 
(seno, coseno, tangente), logaritmo, redondeo hacia abajo (floor) 
y redondeo hacia arriba (ceil), ofreciendo as\'i una amplia gama de herramientas 
matem\'aticas. La inclusi\'on de funciones especializadas como 
la exponencial y el redondeo a un entero mediante la 
función 'round' brinda flexibilidad a los desarrolladores para 
expresar de manera eficiente y concisa operaciones 
matemáticas complejas. Esta sintaxis proporciona una base 
robusta para la manipulación de c\'alculos num\'ericos en 
el contexto de programación de m\'oviles.\newline
La sintaxis del lenguaje es simple y se describe a continuacion. 

\begin{equation*}
\textbf{ - } \textit{ expf }
\text{: Negaci\'on de una expresi\'on flotante (expf).}
\end{equation*}
\begin{equation*}
\textbf{ ( } \textit{ expf }\textbf{ ) }\text{: Agrupaci\'on de una expresi\'on flotante.}
\end{equation*}
\begin{equation*}
\textit{ expf } \textbf{ + } \textit{ expf }\text{: Suma de dos expresiones flotantes.}
\end{equation*}
\begin{equation*}
\textit{ expf } \textbf{ * } \textit{ expf }\text{ Multiplicación de dos expresiones flotantes.}
\end{equation*}
\begin{equation*}
\textit{ expf } \textbf{ - } \textit{ expf }\text{ Resta de dos expresiones flotantes.}
\end{equation*}
\begin{equation*}
\textit{ expf } \textbf{ / } \textit{ expf }\text{ División de dos expresiones flotantes.}
\end{equation*}
\begin{equation*}
\textbf{ sen ( } \textit{ expf } \textbf{ ) }\text{: Seno de una expresión flotante.}
\end{equation*}
\begin{equation*}
\textbf{ cos ( } \textit{ expf } \textbf{ ) }\text{: Coseno de una expresión flotante.}
\end{equation*}
\begin{equation*}
\textbf{ tan ( } \textit{ expf } \textbf{ ) }\text{' expf '): Tangente de una expresión flotante.}
\end{equation*}
\begin{equation*}
\textbf{ log ( } \textit{ expf } \textbf{ ) }\text{: Logaritmo de base 10 una expresión flotante.}
\end{equation*}
\begin{equation*}
\textbf{ floor ( } \textit{ expf } \textbf{ ) }\text{: Redondeo hacia abajo de una expresión flotante.}
\end{equation*}
\begin{equation*}
\textbf{ ceil ( } \textit{ expf } \textbf{ ) }\text{: Redondeo hacia arriba de una expresión flotante.}
\end{equation*}
\begin{equation*}
\textbf{ exp ( } \textit{ expf , expf } \textbf{ ) }\text{: Función exponencial con dos argumentos flotantes.}
\end{equation*}
\begin{equation*}
\textbf{ round ( } \textit{ expf , num } \textbf{ ) }
\begin{aligned}
\text{Redondeo de una expresión flotante con una } \\
\text{cantidad determinada (num) de decimales.}
\end{aligned}
\end{equation*}

\section{Expresiones booleanas}
Las expresiones booleanas se pueden relacionar con los siguientes operadores.

\begin{equation*}
\textbf{true} \text{ y } \textbf{false}\text{: Representan valores booleanos directos.}
\end{equation*}
\begin{equation*}
\textit{expf} < \textit{expf} \text{ y } \textit{expf} > \textit{expf}\text{: Permite comparaciones relacionales entre expresiones flotantes.}
\end{equation*}
\begin{equation*}
\textit{boolexp } \& \textit{ boolexp}\text{: Operador lógico AND para conjunción de expresiones booleanas.}
\end{equation*}
\begin{equation*}
\textit{boolexp } | \textit{ boolexp}\text{: Operador lógico OR para disyunción de expresiones booleanas.}
\end{equation*}
\begin{equation*}
\sim \textit{boolexp}\text{: Operador unario NOT para negación de una expresión booleana.}
\end{equation*}
\begin{equation*}
\textbf{(} \textit{ boolexp } \textbf{)}\text{: Permite agrupar expresiones booleanas para establecer el orden de evaluación.}
\end{equation*}

\section{Comandos b\'asicos y de control de flujo}
En este lenguaje, se emplean comandos fundamentales para la 
manipulación eficiente de variables y el control del flujo de ejecución t\'ipicos
en cualquier lenguaje imperativo.\newline
\begin{equation*}
    \textit{varf} \textbf{ := } \textit{ expf }
\end{equation*}
La asignacion de expresi\'on flotante (expf) a una variable de tipo flotante (varf) se har\'a con el 
operador $:=$. En MoviScript s\'olo se permitir\'a el almacenamiento de valores flotantes 
en las variables. En versiones futuras se podr\'an incorporar variables de punto, de listas de 
expresiones flotantes y de listas de punto.
\begin{equation*}
    \textit{comm} \textbf{ ; } \textit{ comm }
    \text {: La secuenciaci\'on de comandos se realiza con el caracter ';'.}
\end{equation*}
\begin{equation*}
\textbf{ if ( } \textit{boolexp} \textbf{) then } \textit{comm} \textbf{ else } \textit{comm} \textbf{ end }
\end{equation*}
Estructura condicional clasica if-then-else que ejecuta un bloque de comandos u otro seg\'un la evaluación de una expresión booleana.
\newline
\begin{equation*}
\textbf{ while (} \textit{boolexp} \textbf{) do } \textit{comm} \textbf{ end }
\end{equation*}
Establece un bucle que repite la ejecuci\'on de un bloque de
comandos mientras una condici\'on booleana se mantenga verdadera.

\section{Comandos sobre puntos y listas de puntos}
En MoviScript, las instrucciones relacionadas con puntos proporcionan herramientas 
fundamentales para la manipulación espacial y el análisis geom\'etrico. 
Estas instrucciones permiten la opreaciones de puntos en un plano bidimensional 
 como la medición de la distancia entre el origen y un punto específico, 
 as\'i como la determinación del \'angulo que dicho punto forma con el eje x. 
 La capacidad de calcular la distancia y el \'angulo facilita el desarrollo de 
 algoritmos espaciales avanzados, permitiendo a los programadores abordar tareas 
 relacionadas con la ubicaci\'on y la orientación en el contexto específico de 
 dispositivos m\'oviles. \newline
La representaci\'on de un punto en nuestro lenguaje sera un par ordenado de expresiones flotantes.
 \begin{equation*}
point = ( expf , expf )
\end{equation*}
Sobre un punto se pueden realiar los siguientes comandos.
\begin{equation*}
\textbf{dist(}point\textbf{)}
\text{. Establece la distancia entre un punto 
(point) y el origen de coordinadas}
\end{equation*}
\begin{equation*}
    \textbf{rangle(}point\textbf{)}
    \end{equation*}
El comando $rangle$ establece el angulo (en radianes, de 0 a 2\(\pi\)) determinado entre la recta que pasa por el punto 
(point) y el eje x.
\begin{equation*}
    \textbf{gangle(}point\textbf{)}
    \end{equation*}
El comando $gangle$ establece el angulo (en grados, de 0 a 360°) determinado entre la recta que pasa por el punto 
(point) y el eje x.
\begin{equation*}
\textbf{path(}expf , var, [expf] \textbf{)}
\end{equation*}
Con el comando $path$ podremos generar listas de puntos $(x_i,y_i)$ en funci\'on de una 
expresi\'on matem\'atica $(f)$, una variable $(x)$ y una lista de expresiones flotantes
 donde evaluaremos nuestra expresi\'on matem\'atica ($x_i$).
El comando toma tres argumentos:
\begin{itemize}
    \item $expf:$ Es una expresi\'on matem\'atica ($f$) que generara los valores $y_i$. 
    \item $var:$ Es la variable $(x)$ en que se evaluar\'a en la expresi\'on matem\'atica $expf$.  
    \item $[expf]$: Es la lista de valores $x_i$ en la que se evaluar\'a la funci\'on $f$.
\end{itemize}

\section{Comandos de movimientos}
En esta secci\'on describiremos las funciones que brindan la capacidad
 del dispositivo para moverse de manera inteligente y eficiente, 
 ofreciendo a los programadores las herramientas necesarias para la 
 creación de algoritmos de movilidad. Por un lado tenemos herramientas de 
 navegaci\'on b\'asica:
\begin{equation*}
\textbf{fd(} dist , time \textbf{)}
\end{equation*}
Con el comando $Fd$ se indica al m\'ovil el avance lineal de una distancia($dist$, de tipo expresi\'on flotante) en durante en un tiempo determinado($time$, de tipo expresi\'on flotante).
\begin{equation*}
\textbf{bk(} dist , time \textbf{)}
\end{equation*}
Con el comando $Bk$ se indica al m\'ovil el retroceso lineal de una distancia($dist$, de tipo expresi\'on flotante) en durante en un tiempo determinado($time$, de tipo expresi\'on flotante).
\begin{equation*}
\textbf{turn(} ang , time \textbf{)}
\end{equation*}
Con el comando $Turn$ se indica al m\'ovil el giro de un angulo en grados($ang$, de tipo expresi\'on flotante) en durante en un tiempo determinado($time$, de tipo expresi\'on flotante).

\bigskip
El lenguaje, tambi\'en cuenta con instrucciones complejas que permiten elaboraci\'on de algoritmos avanzados:
\begin{equation*}
\textbf{lookat(}point, vel\textbf{)}
\end{equation*}
Con el comando $LookAt$ se indica al m\'ovil el giro en direcci\'on al 
punto ($point$). La coordenada del punto ser\'a en  caracter relativo al propio m\'ovil 
tomando como origen de coordenadas el  m\'ovil y el sentido positivo del eje x ser\'a 
la direcci\'on en la que esta apuntando el m\'ovil al momento de ejecutar la instrucci\'on.
La velocidad de giro ser\'a la indicada como $vel$.
\begin{equation*}
\textbf{goline(}point , vel-lineal , vel-angular \textbf{)}
\end{equation*}
Con el comando $GoLine$ se indica al m\'ovil el giro en direcci\'on al 
punto ($point$) y luego su desplazamiento al punto mencionado. La coordenada 
del punto ser\'a en  caracter relativo al propio m\'ovil 
tomando como origen de coordenadas el  m\'ovil y el sentido positivo del eje x ser\'a 
la direcci\'on en la que esta apuntando el m\'ovil al momento de ejecutar la instrucci\'on.
La velocidad de giro ser\'a la indicada como $vel-angular$ y la velocidad de desplazamiento
ser\'a la indicada como $vel-lienal$.

\begin{equation*}
\textbf{follow(}listpoint , vel-lineal , vel-angular \textbf{)}
\end{equation*}
Con el comando $follow$ se indica que el m\'ovil haga un recorrido determinado
por la lista de puntos ($listpoint = [(x_1,y_1),(x_2,y_2)...(x_n,y_n)]$).
\newline
La velocidad de giro ser\'a la indicada como $vel-angular$ y la velocidad de desplazamiento
ser\'a la indicada como $vel-lienal$.
\begin{equation*}
\textbf{followsmart(}listpoint-allow , listpoint-contingency-allow , vel-lineal , vel-angular \textbf{)}
\end{equation*}                
De igual manera que el comando $follow$, el comando $followsmart$
 indica que el m\'ovil haga un recorrido determinado
por la lista de puntos seguido de las posiciones en las que estar\'a habilitado el m\'ovil 
para circular. La lista de puntos-contingency ser\'a la lista de puntos
($listpoint-allow = def\_obstacles [(x_1,y_1),(x_2,y_2)...(x_n,y_n)], [x_1 ... x_i ... x_n]$) donde
$x_i$ indica la lista de puntos en los que el m\'ovil esta obstaculizado por el medio para ciruclar.
\newline
En este caso, el comando $followsmart$ incorpora informaci\'on del medio. Lo hace mediante la lista de puntos que en este caso
ser\'a una lista de \'indices en los que los puntos obstaculizados por el medio.
\newline
Si  el m\'ovil descure que su camino est\'a impedido por alg\'un obstaculo el lenguaje MoviScript tendr\'a un camino de contingencia 
(relativo al \'ultimo punto de la lista $listpoint$ que logr\'o acceder sin obstaculos). N\'otese que 
esta contingencia es relativo al \'ultimo punto y se utilizar\'a como contingencia de forma 
recursiva, de forma de que si hay obstaculos abordando el camino de contingencia el m\'ovil 
tambi\'en utilizar\'a el camino de contingencia.

\section{Sem\'antica de los comandos}

\textbf{Lookat}
\[
 \frac{p \downarrow (p_x,p_y)}{Lookat(p, v) \to Turn( \frac{180} {\pi} * atan^2(p_x, p_y) , v)}
\]

\textbf{Goline}
\[
 \frac{p \downarrow dist}{Goline (p,v_1,v_2) \to Seq( Lookat(p, v_1) , Fd (\frac{dist}{v_2}, v_2)}
\]

\textbf{TurnAbs}
\[
 \frac{}{TurnAbs(ang, vel) \to Turn\left(\left(\frac{ang - angAct}{v}\right), vel\right)}
\]

\textbf{GolineAbs}
\[
 \frac{p \downarrow (p_x,p_y), p \downarrow dist}{Seq (TurnAbs \left(\frac{180}{\pi} \cdot atan2 (p_x, p_y)\right) \ v_1, Fd(\frac{dist}{v_2}), v_2)}
\]

\textbf{Follow}
\begin{equation}
\frac{}{Follow([], v_1, v_2) \to Skip}
\end{equation}
    
\begin{equation}
\frac{}{Follow([p], v_1, v_2) \to GolineAbs(p, v_1, v_2)}
\end{equation}
    
\begin{equation}
 \frac{p \downarrow (p_x,p_y) , q \downarrow (q_x,q_y)}{Follow(p:q:ps, v_1, v_2) \to {Seq(GolineAbs(p, v_1, v_2), Follow(((q_x - p_x, q_y - p_y) : ps), v_1, v_2))}}
\end{equation}

\textbf{Followsmart}
\setcounter{equation}{0}
\begin{equation}
    \frac{}{FollowSmart([], contingency, v_1, v_2) \to Skip}
    \end{equation}
    
    \begin{equation}
    \frac{}{FollowSmart([p], contingency, v_1, v_2) \to GolineAbs(p, v_1, v_2)}
    \end{equation}
    \begin{equation}
\frac{}{FollowSmart([], contingency, v_1, v_2) \to Skip}
\end{equation}
    
\begin{equation}
\frac{}{FollowSmart([p], contingency, v_1, v_2) \to GolineAbs(p, v_1, v_2)}
\end{equation}


\begin{equation}
\begin{split}
\frac{}{FollowSmart(((p,True):(q,b):xs), contingency, v_1, v_2) \to Seq(GolineAbs(p, v_1, v_2), } \\
FollowSmart(((q_x - p_x, q_y - p_y), b):xs), contingency, v_1, v_2)
\end{split}
\end{equation}
    
\begin{equation}
    \begin{split}
\frac{p \downarrow (p_x,p_y) , q \downarrow (q_x,q_y)}{FollowSmart(((p, False):(q,b):xs), [], v1, v2) \to} \\
FollowSmart((((q_x - p_x, q_y - p_y), b):xs), [], v_1, v_2)
    \end{split}
\end{equation}
    
\begin{equation}
    \begin{split}
\frac{p \downarrow (p_x,p_y) , c \downarrow (c_x,c_y)}{FollowSmart((((p, False):xs), ((c,b):cs), v_1, v_2)) \to } \\
FollowSmart((((c_x + p_x, c_y + p_y), b):cs \mathop{+\!\!+} xs), ((c,b):cs), v_1, v_2)
    \end{split}
\end{equation}

\section{Posibles pr\'oximas versiones}

En futuras versiones, MoviScript podr\'a tomar informaci\'on del medio con un sensor frontal del m\'movil. 
Tambi\'en se podr\'a compilar MoviScript a c\'odigo 
$NQC$\footnote{Not Quite C es un lenguaje simple que sintaxis similar a C utilizado para programar 
la \'inea rob\'otica de LEGO denominada Maindstorms. Para m\'as informaci\'on visitar la 
documentaci\'on: https://bricxcc.sourceforge.net/nqc/}
 y de esta forma podr\'a ser ejecutado en un m\'ovil construido 
 en $LEGO MINDSTORMS$.

\end{document}
